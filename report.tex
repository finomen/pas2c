\documentclass[11pt,a4paper,oneside]{report}
\usepackage [height=25cm,a4paper,hmargin={2cm,2cm}]{geometry}
\usepackage[T2A]{fontenc}
\usepackage[utf8]{inputenc}
\usepackage[english,russian]{babel}
\usepackage{url}
\usepackage{graphicx}
\usepackage{indentfirst}
\usepackage{misccorr}
\usepackage{amsmath}
\usepackage{amssymb}

\usepackage{listings}
\lstloadlanguages{C++} %Загружаемые языки

\lstset{extendedchars=true, %Чтобы русские буквы в комментариях были
        commentstyle=\it,
        stringstyle=\bf,
        language=Java, %Язык по умолчанию
        breaklines=true,
        breakatwhitespace=true,
        belowcaptionskip=5pt}

\renewcommand{\epsilon}{\varepsilon}
\renewcommand{\thesection}{\arabic{section}}

\begin{document}
\title{Лабораторная работа №3 \endgraf << Использование автоматических генераторов анализаторов >>}
\author{Николай Фильченко, гр. 3539}

\begin{titlepage}

\begin{center}


% Upper part of the page   
\textsc{Санкт-Петербургский государственный университет
информационных технологий, механики и оптики \\
Факультет информационных технологий и программирования \\
Кафедра <<Компьютерные Технологии>>}\\[8.5cm]


% Title
\textsc{\Large Лабораторная работа №3. Использование автоматических генераторов анализаторов}\\[0.5cm]

\textsc{\Large Вариант 3. Перевод с Паскаля на Си}\\[3cm]


% Author and supervisor
\textsc{Николай Фильченко, гр. 3539}

\vfill

% Bottom of the page
{\large \today}

\end{center}

\end{titlepage}

\setcounter{page}{2}

\tableofcontents

\newpage

\section{Постановка задачи}

Транслировать программу с языка \emph{Pascal} на язык \emph{C++}. За основу взято описание грамматики языка 
из Standard Pascal -- User Reference Manual", которая адаптирована для использования в boost.spirit 
(убрана левая рекурсия, немного упрощены правила с опциональными частями и последовательности), добавлены
константные выражения.

Не реализованы следующие кострукции
\begin{itemize}
\item Типизированные файлы;
\item Определение составных типов;
\item Массивы нетривиальных типов;
\item Поля в структурах нетривиальных типов;
\item Множества;
\item Оператор case;
\item Вложенные неполные условные операторы;
\end{itemize}

Возведение в степень и массивы транслируются, но не имеют реализации на \emph{C++}.

\section{Построение грамматики}

Построена атрибутно-управляемая грамматика, распознающая указанное подмножество языка \emph{Pascal}
и выполняющая указанные преобразования.

\lstinputlisting{grammar_ok.cpp}

Также небольшой вспомогательный заголовочный файл

\lstinputlisting{pascal.h}

\section{Тестирование}

Небольшой тест:

\lstinputlisting{test.pas}

Результат работы транслятора:

\lstinputlisting{test.cpp}

\end{document}